\documentclass{beamer}

\title{A Kochen-Specker system has at least 21 vertices}

\author{Sander Uijlen \and Bas Westerbaan}
\institute{Radboud Universiteit Nijmegen}

% we don't want those funny buttons down right.
\setbeamertemplate{navigation symbols}{}

\begin{document}

\begin{frame}
    \titlepage
\end{frame}

\begin{frame}{}
        A \alert{Kochen-Specker system} $S$ is a finite set
        of points on the (open) northern hemisphere,
        such that there is no~$010$-coloring; that is: there is no~
        $\{0,1\}$-valued coloring with
        \begin{enumerate}
            \item
                three pairwise orthogonal points are assigned~$(1,0,0)$,
                        $(0,1,0)$ or~$(0,0,1)$ and
            \item
                two orthogonal points are not assigned~$(1,1)$.
                \\~\\
        \end{enumerate}
    \begin{tabular}{ll}
        point & $\sim$ direction of magnetic field
                    in measurement of SPIN-1 \\
        coloring & $\sim$ non-contextual deterministic theory
    \end{tabular}
    \pause
    \begin{theorem}[Kochen-Specker]
        There is a Kochen-Specker system.  Thus:
        there is no non-contextual deterministic theory
        predicting the measurement
        of a SPIN-1 particle.
    \end{theorem}
\end{frame}

\begin{frame}{The smallest Kochen-Specker system?}
    \begin{tabular}{lrl}
        Kochen-Specker & 1975 & $\leq 117$ \\
        Penrose, Peres (indep.) & 1991 & \onslide<2->{$\leq 33$} \\
        Conway & $\sim$ 1995 & \onslide<3->{$\leq 31$} \\
               & & \\
        \onslide<6->{U\&W}       & \onslide<6->{july?}&
                \onslide<6->{$\geq 22$ or $=21$} \\
        \onslide<5->{U\&W}       & \onslide<5->{may}& \onslide<5->{$\geq 21$} \\
                Arends, Wampler, Oauknine & 2009 &\onslide<4->{$\geq 18$} 
    \end{tabular}
\end{frame}

\begin{frame}{It is a problem about graphs}
    
\end{frame}

\end{document}

% vim: ft=tex.latex
