\documentclass[a4paper]{article}

\usepackage{algorithm2e}
\usepackage{hyperref}
\usepackage{paralist}
\usepackage{graphicx}
\usepackage{authblk}
\usepackage{amsmath}
\usepackage{wrapfig}
\usepackage{amsthm}

\usepackage{draftwatermark}

% TODO replace 21 with 22
\newcommand{\keyword}[1]{\textbf{#1}}

\newcounter{main}
\newtheorem{prop}[main]{Proposition}
\newtheorem{comp}[main]{Computation}
\newtheorem{cor}[main]{Corollary}
\newtheorem{thm}[main]{Theorem}
\newtheorem{lem}[main]{Lemma}
\newtheorem{fact}[main]{Fact}
\theoremstyle{definition}
\newtheorem{dfn}[main]{Definition}
\newtheorem*{spin}{SPIN Axiom \cite{ck09}}
\theoremstyle{remark}
\newtheorem{rem}[main]{Remark}

\title{A Kochen-Specker system has at least 21 vertices\\
        {\small (extended abstract)}}

\author{Sander Uijlen}
\author{Bas Westerbaan}

% TODO will we keep these @cs.ru.nl addresses?
\affil{{\small Institute for Computing and Information Sciences\\
        Radboud Universiteit Nijmegen}\\
   \{\texttt{suijlen},\texttt{bwesterb}\}\texttt{@cs.ru.nl}}

\begin{document}

\maketitle
\thispagestyle{empty}

\begin{abstract}
    At the heart of the Conway's Free Will theorems and Kochen and Specker's
        argument against non-contextual hidden variable theories
    is the existence of a Kochen-Specker (KS) system:
    a set of points on the sphere
    that has no~$\{0,1\}$-coloring such that
    at most one of two orthogonal points are colored~$1$
    and of three pairwise orthogonal points exactly one
    is colored~$1$.
    In public lectures, Conway encouraged the search for small
    KS systems.
    At the time of writing, the smallest known
    KS system has 31 vectors.  

    Arends, Ouaknine and Wampler have shown that a KS system has at least
    18 vectors, by reducing the problem to the existence of graphs
    with a topological embeddability and non-colorability property.
    The bottleneck in their search
    proved to be the sheer number of graphs on more than~$17$
    vertices and deciding embeddability.

    Continuing their effort, we prove a restriction on the class of graphs
    we need to consider and develop a more practical decision procedure for
    embeddability to improve the lower bound to 21.
\end{abstract}

\clearpage
    
\section{Introduction}

% TODO whose idea is this orthogonality graph
\subsection{The experiment}

% TODO correct description of experiment
Consider the following experiment.  Shoot a deuterium atom,
or any other spin-1 particle,
along, say: the x-axis, through a inhomogeneous magnetic field.
Depending on the direction of the magnetic field,
the particle will move undisturbed
or deviate.

Quantum Mechanics only predicts the probability, given the configuration
of the field, whether the particle will deviate.
Its probabilistic prediction has been thoroughly tested.
One wonders: is there a deterministic non-contextual theory predicting the
outcome of this experiment?

Kochen and Specker proved that such a theory cannot satisfy:
\begin{spin}
    Given three pairwise orthogonal directions.
    In exactly one of the directions, the particle will not deviate.
\end{spin}
Their argument is based on the existence of a Kochen-Specker system.
\begin{dfn}
    A \keyword{Kochen-Specker (KS) system} is
    a finite set of points on the sphere\footnote{
            We define KS systems to be three dimensional,
            as in the original proof of Kochen and Specker.
            Later, higher dimensional systems have been studied.
            See, for instance~\cite[p.~201]{qtcm}.
        }
    for which each pair is not antipodal and
    there is no~\keyword{010-coloring}.
    A $010$-coloring is a~$\{0,1\}$-coloring of the points such that
        \footnote{
                In other papers, like \cite{aow11},
                the~$0$ and~$1$ are swapped; they consider 101-colorings.
                These colorings are of course equivalent and the
                difference arises from considering either squared
                spin measurements $S^2$, or $1-S^2$.
                %Then the colors correspond directly to the squared measurement
               	%outcome of the spin.
               	}
    \begin{enumerate}
        \item
            no pair of orthogonal points are both colored~$1$ and
        \item
            of three pairwise orthogonal points exactly one is colored~$1$;
            or alternatively: they are colored~$0$, $1$ and~$0$ in some order.
    \end{enumerate}
\end{dfn}
% TODO shorten this proof.
A point on the sphere obviously corresponds to a direction in space.
Because of this, the term point, vector and direction
can be used interchangeably. Antipodal points correspond to opposite
vectors and these span the same direction in space.

Suppose there is a KS system and a non-contextual deterministic theory satisfying
the SPIN Axiom.
Then we color a point of this system~$0$,
whenever this theory predicts that the particle will deviate
if the spin is measured in the direction corresponding to that
point, and~$1$ otherwise.
Given two orthogonal points of the system,
we can find a third point orthogonal to both of them.
The SPIN axiom implies exactly one of them is colored~$1$, so they
cannot both be colored~$1$.
Similarly, given three pairwise orthogonal vectors in the system,
the SPIN axiom implies exactly one of them is colored~$1$.
Hence there would be a 010-coloring of the KS system, quod non.
Therefore a deterministic non-contextual theory cannot satisfy the
SPIN Axiom.

The KS system proposed by Kochen and Specker contained 117 points\cite{ks}.
Penrose and Peres\cite{peres} independently found a smaller system of 33 points.
The current record is the 31 point system of Conway\cite[p.~197]{qtcm}.
As pointed out by \cite{c00,aow11}, finding small KS systems
is of both theoretical and practical interest.
In public lectures, Conway himself, stressed the search for small KS
systems.\cite{OC}

% TODO refer to non-3d systems
\subsection{Overview}
In \cite{aow11} Arends, Ouaknine and Wampler (AOW) give a computer aided proof
that a KS system must have at least 18 vectors.  We improve their lower bound
and show that a KS system must have at least 21 vectors.

First, in Subsection~\ref{sec:ksgraphs},
we repeat a part of AOW's work, in particular the reduction of
KS systems to graphs.
The bottleneck of their search was the sheer number of graphs
and the deciding whether such graphs are embeddable.

In Section~\ref{sec:ilb},
we improve upon their reduction,
to cut down the number of graphs to consider drastically,
and state the results of our main computation.
Finally, in Section~\ref{sec:emb},
we describe our practical embeddability test.

\subsection{Kochen-Specker graphs}
\label{sec:ksgraphs}
We follow \cite{aow11} and reduce the search for Kochen-Specker systems
to the search of a certain class of graphs.
First note that in a Kochen-Specker system we may replace a point with its
antipodal point.  They are both orthogonal to the same points and hence
the non-010-colorability is preserved.
Therefore, we may assume antipodal points are identified on the sphere.
That is: a Kochen-Specker system is a finite subset of the projective plane
that is not 010-colorable.
\begin{dfn}
Given a finite subset~$S$ of the projective plane
(or equivalently, a finite subset of the northern
hemisphere without equator).
Define its \keyword{orthogonality graph}~$G(S)$ as follows.
The vertices are the points of~$S$.
Two vertices are joined by an edge, if their corresponding points
are orthogonal.
\end{dfn}
\begin{dfn}
A graph~$G$ is called~\keyword{embeddable},
if it occurs as a subgraph of an orthogonality graph;
that is: if there is a finite subset~$S$ of the projective plane,
such that~$G \leq G(S)$.
\end{dfn}
% TODO add remark on subgraph definition?
\begin{dfn}
A graph is~\keyword{010-colorable}
if there is a~$\{0,1\}$-coloring, such that
\begin{enumerate}
\item
for each triangle there is exactly one vertex that is colored~$1$ and
\item
adjacent vertices are not both colored~$1$.
\end{enumerate}
\end{dfn}
\begin{dfn}
A \keyword{Kochen-Specker graph}
is a embeddable graph that is not 010-colorable.
\end{dfn}
It is an easy, but important, consequence of the definitions that:
\begin{fact}
    A finite subset~$S$ of the projective plane
    is a Kochen-Specker system,
    if and only if its orthogonality graph~$G(S)$
    is Kochen-Specker.
\end{fact}
To prove there is no Kochen-Specker system on~$17$ points,
it would be sufficient to enumerate all graphs on~$17$ vertices
and check these are not 010-colorable or not embeddable.
However, this is infeasible as there are
already~${\sim}10^{26}$ non-isomorphic
graphs on~$17$ points.\cite{oeisA000088}
Luckily, we can restrict ourselves to certain classes of graphs.
\begin{prop}[\cite{aow11}]
    An embeddable graph is squarefree.
    That is: it does not contain the square as a subgraph.
    \footnote{Some authors call a graph squarefree if it does not
        contain the square as induced subgraph.
        For them the complete graph on four vertices is squarefree.
        We follow~Weisstein\cite{sf-weisstein} and~Sloane\cite{sf-sloane} and
        call a graph squarefree if it does not
        contain the square as subgraph.
        For us the complete graph on four vertices is not squarefree.}
\end{prop}
\begin{proof}
    Given two non antipodal points~$a\neq b$.
    Consider the points orthogonal to~$a$.
    This is a great circle.
    The points orthogonal to~$b$ is a different great circle.
    They intersect in precisely two antipodal points.
    Hence, if~$c$ and~$d$ are both orthogonal to~$a$ and~$b$,
    then~$c$ and~$d$ are equivalent.
    Therefore, an embeddable graph cannot contain a square.
\end{proof}
The squarefreeness is a considerable restriction.  There are
only~${\sim}10^{10}$ non-isomorphic squarefree graphs on~$17$
vertices.\cite{sf-sloane}
We can restrict ourselves to connected graphs.
\begin{prop}[\cite{aow11}]\label{prop:ks-conn}
    A minimal Kochen-Specker graph is connected.
\end{prop}
\begin{proof}
    Suppose~$G$ is a non-connected Kochen-Specker graph.
    Then one of its components is not 010-colorable.
    As a subgraph of an embeddable graph, is embeddable,
    this component is embeddable as well.
    Hence it is a smaller connected Kochen-Specker graph.
\end{proof}
The gain, however, is small.
There are only~${\sim}10^9$ non-isomorphic squarefree graphs on~$17$
vertices that are not connected.
In our computations, checking for connectedness
required more time than would be gained by reducing the number of graphs.

We have verified the main result of \cite{aow11}:
\begin{comp}
There is a unique non-010-colorable squarefree connected graph on~$17$ or less
vertices:
\begin{center}
\includegraphics[width=50mm]{graphs/c17.jpg}
\end{center}
It is not embeddable, as the graph in Figure~\ref{fig:unemb-10-2}
is an unembeddable subgraph.  For our proof,
see Proposition~\ref{prop:unemb-10-2}.
Hence a Kochen-Specker
system has at least 18 points.
\end{comp}

\section{An improved lower bound}
\label{sec:ilb}
Continuing the effort of Arends, Ouaknine and Wambler,
we consider another restriction.
\begin{prop}
    A minimal Kochen-Specker graph has minimal vertex-order three.
\end{prop}
\begin{proof}
    Given a Kochen-Specker graph~$G$.
    Suppose~$v$ is a vertex with order less than or equal~$2$.
    Let~$G'$ be~$G$ with~$v$ removed.
    Clearly~$G'$ is embeddable.
    Suppose~$G'$ is 010-colorable.
    Then we can extend the coloring to a coloring of~$G$ as follows.
    If~$v$ is adjacent to only one or no vertex,
    then we can color~$v$ with~$0$.
    Suppose~$v$ is adjacent to two vertices, say~$w$ and~$w'$.
    If one of~$w$ or~$w'$ is colored~$1$, we can color~$v$ with~$0$.
    If both~$w$ and~$w'$ are colored~$0$, we can color~$v$ with~$1$.
    This would imply~$G$ is 010-colorable, quod non.
    Therefore~$G'$ is a smaller
    Kochen-Specker graph, which contradicts minimality.
\end{proof}
There are only~${\sim}10^7$
squarefree non-isomorphic graphs on 17 vertices with minimal vertex order 3.
Even though Arends, Ouaknine and Wampler
note this restriction once,
surprisingly, they did not restrict their graph enumeration
to graphs with minimal vertex order 3.

We continue with a strengthening of Proposition~\ref{prop:ks-conn}.
\begin{prop}\label{prop:ks-biconn}
A minimal Kochen-Specker graph is biconnected,
that is: removing any single edge leaves the graph connected.
\end{prop}
We need some preparation, before we can prove this Proposition.
\begin{dfn}
Given a graph~$G$ and a vertex~$v$ of~$G$.
We say, \keyword{$v$ has fixed color~$c$ (in~$G$)},
if~$G$ is~$010$-colorable
and for every~$010$-coloring of~$G$,
the vertex~$v$ is assigned color~$c$.
\end{dfn}
We are interested in these graphs because of the following observation.
\begin{lem}
If there is an embeddable graph~$G$ on~$n$ vertices with a vertex
with fixed color~$1$,
then there is a Kochen-Specker graph on~$2n$ vertices.
\end{lem}
\begin{proof}
Let~$G$ be a graph and~$v$ a vertex of~$G$ with fixed color~$1$.
Consider two copies of the graph~$G$.
Connect the two instances of~$v$ with an edge.
Call this graph~$G'$.
Clearly, $G'$ is not~$010$-colorable.

We need to show~$G'$ is embeddable.
Given an embedding~$S$ of~$G$.
We may assume that the point in~$S$ corresponding to~$v$
is the north pole.
Furthermore, we may assume that there is no point on the $x$-axis,
by rotating points along the north pole.
Let~$S'$ be~$S$ rotated~$90$ degrees along the~$y$-axis.
Some points of~$S$ and~$S'$ might overlap.
That is: there might be a point~$s$ in~$S$ and~$s'$ in~$S'$
that are equal or antipodal.
Observe that if no points of~$S'$ and~$S$ overlap,
then~$S \cup S'$ is an embedding of~$G'$.

Suppose there are points in~$S'$ and~$S$
that overlap.
Note that the north pole (and south pole) is not in~$S'$. 
Let~$S''$ be~$S'$ rotated along the north pole at some angle~$\alpha$.
There are finitely many angles such that there are overlapping points.
Thus there is an angle such that~$S \cup S''$ is an embedding of~$G'$.
\end{proof}
Unfortunately, these graphs are not small.
\begin{comp}\label{comp:bic1}
Let~$F^{3,1}_n$ denote the number of connected graphs
with minimal vertex-order~$3$
on~$n$ vertices
with a vertex with fixed color~$1$.
Then:
\begin{center}
\begin{tabular}{l|lll}
    $n$ & $\leq 13$
        & $14$
        & $15$ \\
    \hline
    $F^{3,1}_n$ & $0$
        & $4$
        & $59$
\end{tabular}
\end{center}
Using the methods of Section~\ref{sec:emb},
we have determined these graphs unembeddable.
\end{comp}
We need one final computational result.
\begin{comp}\label{comp:bic2}
There are no connected graphs
on less than~$15$ vertices
with minimal vertex-order~$2$
with a vertex with fixed color~$0$. 
\end{comp}
We are ready to prove that a minimal Kochen-Specker graph
is biconnected.
\begin{proof}[Proof of Proposition~\ref{prop:ks-biconn}]
Suppose we are given a connected but not biconnected graph~$G$.
Then we can decompose
this graph into its biconnected components.  A biconnected components
is a maximal subgraph that is biconnected.
We can consider the graph of biconnected components~$B(G)$:
two biconnected components are adjacent if there is an edge
between a vertex in one component and a vertex in the other.
Note that there can at most be one edge between 
the vertices of biconnected components.
$B(G)$~does not contain loops: if it would
then the union of the biconnected components in the graph,
would be itself biconnected.  Thus~$B(G)$ is a tree.

Furthermore suppose~$G$ is a minimal Kochen-Specker graph.
Now consider a leaf~$A$ in the biconnected component graph~$B(G)$.
That is, $A$ is a biconnected component that is connected to only
one other biconnected component.
Let~$B$ be the remainder of the graph.
There is exactly one pair~$(a,b)$ with~$a$ in~$A$ and~$b$ in~$B$
and~$a$ adjacent to~$b$.

Note that~$A$ and~$B$ are~$010$-colorable.
Suppose that~$a$ does not have fixed color~$1$ in~$A$.
Then there is a~$010$-coloring of~$A$
which assigns~$0$ to~$a$.
But then~$G$ is~$010$-colorable, quod non.
Thus~$A$ is an embeddable graph with a vertex with fixed color~$1$.
Similarly~$B$ is an embeddable graph with a vertex with fixed color~$1$.
We may assume~$G$ is two copies of~$A$ connected at the
copies of the vertex with fixed color~$1$.

The graph~$G$ has minimal vertex-order~$3$.
Thus every vertex of~$A$ has minimal order~$3$ except for~$a$,
which has minimal order~$2$.
Suppose~$a$ has minimal order~$3$ in~$A$.
Then by Computation~\ref{comp:bic1},
the graph~$A$ has at least~$16$ vertices.
Thus~$G$ has at least~$32$ vertices, quod non.

Thus~$a$ must have order~$2$ in~$A$.
Let~$c,d \in A$ be the vertices connected to~$a$ in~$A$.
For~$a$ to have fixed color~$1$ in~$A$,
the vertices~$c$ and~$d$ must have fixed color~$0$ in~$A$.
By Computation~\ref{comp:bic2},
the graph~$A$ has at least~$15+1$ vertices.
Thus~$G$ has at least~$32$ vertices. Contradiction.
\end{proof}
We believe a minimal KS graph is also triconnected.
However, we did not find a proof.

Although these restrictions are theoretically pleasing,
they seem to be of little use as a practical restriction.
Concerning excluding unconnected graphs:
% TODO Add argument or computation for {2,3}-connected
% TODO note that restricting to (bi/tri)-connected is a waste of time
% TODO add graph 
\begin{comp}
    There are five non-isomorphic minimal
    squarefree connected graphs
    with minimal vertex order 3 and they have 10 vertices.
\end{comp}
\begin{cor}
    Any unconnected
    squarefree graph with minimal vertex order 3
    has at least 20 vertices, for it has two connected components,
    each with at least 10 vertices.
    With 20 vertices, there are exactly 25 of these.
\end{cor}
This justifies, at this stage, not checking for connectedness.
Similarly, we believe there are very few connected but not biconnected graphs.

% TODO is core x years a proper expression?
Now we can state our main computation.
\begin{comp}
    Let~$C_n$ denote the number of non-010 colorable squarefree
    graphs with minimal vertex order 3 on~$n$ nodes.  Then:

    \begin{center}
    \begin{tabular}{l|lllll}
        $n$ & $\leq 16$
            & $17$
            & $18$
            & $19$
            & $20$ \\
        \hline
        $C_n$ & $0$
            & $1$
            & $2$
            & $19$
            & $441$
    \end{tabular}
    \end{center}

    All these 463 graphs are not embeddable.
    See Computation~\ref{comp:unemb20}.
\end{comp}
The computation took roughly a week on a 64-core Opteron 6276.
It was executed as follows.
We enumerated all squarefree graphs with minimal vertex
order 3 on less than or equal~$20$ vertices,
using the~\texttt{geng} util of the nauty software package,
which uses the isomorphism-free exhaustive generation
method of McKay\cite{geng}.
The output of~\texttt{geng}, we passed through
a custom heuristic backtracker written in~C++
to decide 010-colorability of these graphs.

\section{Embeddability}
\label{sec:emb}
\begin{wrapfigure}{r}{0.40\textwidth}
\begin{center}
\includegraphics[width=50mm]{graphs/unemb-10-2.jpg}
\end{center}
\caption{One of the two minimal
        non-embeddable graphs
\label{fig:unemb-10-2}}
\end{wrapfigure}

Our computation has yielded a few hundred non-010-colorable graphs.
If we show one of them is embeddable, we have found a new KS system.
If we demonstrate all of them are not embeddable, we have
proven a lower bound on the size of a minimal KS system.

In~\cite{aow11}, Arends, Wampler and Ouaknine discuss several
computer-aided methods
to test embeddability of a graph.  None of these methods could decide
for all graphs considered, whether they were embeddable or not.
% TODO give a small description of their deficiencies
% TODO give intermediate computation.

We propose a new method,
which for all graphs we considered,
could decide
whether they were embeddable or not.
First we give a pen-and-paper example.
\begin{prop}\label{prop:unemb-10-2}
The graph in Figure~\ref{fig:unemb-10-2}
is not embeddable.
\end{prop}
\begin{proof}
Suppose it is embeddable.
Consider~$p_1$.
It is orthogonal to both~$a$ and~$v$.
$a$ and~$v$ are not collinear,
hence~$p_1$ must be collinear to~$v \times a$,
the cross-product of~$v$ and~$a$.
Similarly, $p_2$ is collinear to~$v \times p_1 = v \times (v \times a)$.
Continuing in this fashion,
we see that
\begin{equation}\label{eq:ue1}
    a \text{ is collinear to }
    x \times (x \times( w \times (w\times (v \times (v \times a))))).
\end{equation}
Now, we may assume that~$z=(0,0,1)$ and $x=(1,0,0)$.
Thus: $v=(v_1,v_2,0)$;
$w = (w_1,w_2,0)$
and $a = (0, a_2,a_3)$ for some~$-1 \leq v_1,v_2,w_1,w_2,a_2,a_3 \leq 1$,
with~$v_1^2+v_2^2 = 1$; $w_1^2+w_2^2=1$ and~$a_2^2 + a_3^2=1$.
Now, \eqref{eq:ue1} becomes:
\begin{equation*}
\begin{pmatrix}
0\\
a_2\\
a_3\\
\end{pmatrix}
\text{ is collinear to }
\begin{pmatrix}
0\\
-a_2 v_1w_2 (v_1w_1 + v_2w_2) \\
-a_3 (v_1^2 w_1^2 + v_1^2 w_2^2 + v_2^2w_1^2 + v_2^2w_2^2)\\
\end{pmatrix}.
\end{equation*}
And therefore
\begin{align*}
v_1w_2 \left<v,w\right> & =
v_1w_2 (v_1w_1 + v_2w_2) \\
& = v_1^2 w_1^2 + v_1^2 w_2^2 + v_2^2w_1^2 + v_2^2w_2^2 \\
& = (v_1^2 + v_2^2)w_1^2 + (v_1^2 + v_2^2)w_2^2 \\
& = w_1^2 + w_2^2 = 1.
\end{align*}
Since~$v$ and~$w$ are not collinear,
we have by Cauchy-Schwarz $|\left<v,w\right>| < 1$.
Recall~$|v_1|, |w_2| \leq 1$.
Thus: $|v_1w_2\left<v,w\right>| < 1$.
Contradiction.
\end{proof}

% TODO Can we prove: if G is embeddable, then we can add any new vertex of
%      order 2 and G is still embeddable?

In the previous proof, we fixed, without loss of generality, the position
of a few vertices.  Then we derived cross-product expressions for the
remaining vertices.  Finally, we find an equation relating some of
the cross-product expressions and show it is unsatisfiable.
We can automate this reasoning as follows.

\begin{algorithm}
    \While{there are unassigned vertices}{
        pick an unassigned vertex~$v$\;
        assign~$V(v)=v$\;
        mark~$v$ as free\;
        \While{there are unassigned vertices adjacent to two
                different assigned vertices}{
            pick such a vertex~$w$ adjacent to the assigned ~$w_1$ and~$w_2$\;
            assign~$V(w)=V(w_1) \times V(w_2)$\;
            mark edges~$(v,w_1)$ and~$(v,w_2)$ as accounted for\;
        }   
    }
    \For{each pair of vertices~$(v_1, v_2)$}{
        \If{$(v_1,v_2)$ is not an edge} {
            record requirement:~``$V(v_1)$ is not collinear to $V(v_2)$''\;
        }
    }
    \For{each edge~$(v_1,v_2)$ not accounted for}{
        record requirement:~``$V(v_1)$ is orthogonal to~$V(v_2)$''\;
    }
\end{algorithm}
At two points in the algorithm, there is a choice which vertex to pick.
Depending on the vertices chosen, the number of recorded requirements
and free points may significantly vary. By considering all possible choices,
one can find the one with least free points.

The requirements can be mechanically converted
to a formal sentence
in the language of the real numbers.
This sentence is true if and only if the graph is embeddable.
Famously, Tarski proved\cite{tarski}
that such sentences are decidable.
His decision procedure has an impractical complexity.
However, its practical value has been improved
by, for instance, the method of cylindrical algebraic decomposition\cite{qecad}.
We have used the redlog\cite{redlog} package of the reduce algebra
system, which implements a variant of Tarski's quantifier elimination.
In Appendix~\ref{adx:unemb-10-2} the Reader can find
the reduce script generated mechanically for the graph
in Figure~\ref{fig:unemb-10-2}.

Different assignments give different sentences.  In our tests,
some assignments would yield sentences that were decided within milliseconds,
whereas another assignment with less free vertices would
yield a sentence that could not be decided (directly).
Therefore, when determining embeddability of a graph,
we try several assignments in parallel.

In this way, there were still a few (010-colorable) graphs of which
we could not decide embeddability.
By hand, we determined these graphs to be embeddable.
We adapted the algorithm, as to guess
for some assignments the position of one of the vectors.
If the corresponding
sentence turns out false, we know nothing.  However,
if the sentence is true, we know the graph is embeddable.

With this method, we have decided in an hour the embeddability
of every squarefree graph with minimal vertex order three of 13
vertices of less.
In particular:
\begin{comp}\label{comp:unemb20}
    Every squarefree graph of minimal vertex order three
    that is not 010-colorable
    of order less than or equal to 20
    contains, as a subgraph, one of the following three graphs:
    \begin{center}
        \includegraphics[width=120mm]{graphs/unemb-base-20.jpg}
    \end{center}
    These three graphs are unembeddable.  The left and middle graph
    are the only minimal unembeddable squarefree graph.
\end{comp}
For the first graph, we have proven directly that it is unembeddable.
See Proposition~\ref{prop:unemb-10-2}.
For the second graph, we also have a similar direct proof. The third graph is shown to not be embeddable using our algorithm.

\section{Conclusion and future research}
Arends, Ouaknine and Wampler struggled with two problems:
enumerating candidate graphs of less than 31 vertices
and testing their embeddability.
We have verified most of their computations.
Then we enumerated all candidate graphs
up to and including 20 vertices.
Furthermore, we have proposed a new decision procedure,
which was able to decide embeddability
for all candidate graphs we found.
Therefore, we demonstrate: a Kochen-Specker system must have at least
21 points.
At the time of writing,
we are computing
whether there is a KS system of 21 vertices.\footnote{
The authors have a wager whether there is a minimal KS system of less
than 25 vertices.}

Enumerating all candidate graphs of less than 31 vertices
is computationally infeasable.
To bridge the enormous the gap between 21 and 31,
requires a new insight.
For instance: another restriction on which graphs to consider.

The Reader, interested in pursuing this line of research,
is encouraged to read the master thesis\cite{a09} of Arends,
in which he discusses in detail several other
properties that a minimal KS system must enjoy, as well as
some failed attempts.

\section{Acknowledgments}
We wish to thank the following for their generous contribution to the
distributed computation:
    the Digital Security group, Intelligent Systems group
    and the C\&CZ service of the Radboud University;
    Wouter Geraedts and
    Jille Timmermans.

We are grateful to prof.~McKay for discussing
the feasibility of certain graph restrictions.

\clearpage
\appendix
\section{Example reduce script}\label{adx:unemb-10-2}
The following is (a part of) the reduce script mechanically generated
to prove the non-embeddability of the graph of Figure~\ref{fig:unemb-10-2}.
The algorithm choose a different assignment, than we did in the
proof of Proposition~\ref{prop:unemb-10-2}.
As free points it picked, in order,~$z$, $x$, $v$ and~$p_3$.
The point~$w$ is assigned~$p_3 \times z$.
\begin{verbatim}
load_package redlog;
rlset R;

procedure d(x,y);
    (first x) * (first y) +
    (second x) * (second y) +
    (third x) * (third y);

procedure k(x,y);
    {(second x)*(third y) - (third x)*(second y),
     (third x)*(first y) - (first x)*(third y),
     (first x)*(second y) - (second x)*(first y)};

v0c1 := 1; v0c2 := 0; v0c3 := 0;
v1c1 := 0; v1c2 := 1; v1c3 := 0;

v0 := {v0c1, v0c2, v0c3}; 
v1 := {v1c1, v1c2, v1c3}; 
v2 := {v2c1, v2c2, v2c3}; 
v3 := {v3c1, v3c2, v3c3}; 
v2c1 := 0;
neq0 := k(v0,k(v3,v1)); 
\end{verbatim}
\begin{center} \emph{(snip)} \end{center}
\begin{verbatim}
neq29 := k(k(k(k(v3,v1),v1),v2),k(k(v3,v0),v3)); 
phi := 
       (first neq0 neq 0 or
        second neq0 neq 0 or
        third neq0 neq 0) and 
\end{verbatim}
\begin{center} \emph{(snip)} \end{center}
\begin{verbatim}
       (first neq29 neq 0 or
        second neq29 neq 0 or
        third neq29 neq 0) and 
       d(v2,v0) = 0 and 
       d(k(k(v3,v0),v3),k(k(k(k(v3,v1),v1),v2),v2)) = 0 and 
        true;
rlqe ex(v3c3,
     ex(v3c2,
     ex(v3c1,
     ex(v2c3,
     ex(v2c2,phi)))));
\end{verbatim}


% attribute mckay for answering questions?
% TODO attribute geng
\clearpage
\bibliography{main}{}
\bibliographystyle{plain}


\end{document}

% vim: ft=tex.latex
