\documentclass[a4paper]{article}

\usepackage{paralist}
\usepackage{authblk}
\usepackage{amsthm}

\newcounter{main}
\newtheorem{prop}[main]{Proposition}
\newtheorem{thm}[main]{Theorem}
\newtheorem{lem}[main]{Lemma}
\theoremstyle{definition}
\newtheorem{dfn}[main]{Definition}
\theoremstyle{remark}
\newtheorem{rem}[main]{Remark}

% TODO this suggests only one result: the improved lower bound
%      reference to other results?
\title{A Kochen-Specker system has at least 22 vertices}

\author{Sander Uijlen}
\author{Bas Westerbaan}

% TODO will we keep these @cs.ru.nl addresses?
\affil{Institute for Computing and Information Sciences\\
       Radboud Universiteit Nijmegen\\
   \{\texttt{suijlen},\texttt{bwesterb}\}\texttt{@cs.ru.nl}}

\begin{document}

\maketitle

\begin{abstract}
    At the heart of the Conway's Free Will theorems and Kochen-Specker's
        argument
    is the existence of a Kochen-Specker (KS) system:
    a set of points on the sphere,
    that has no~$\{0,1\}$-coloring such that
    at most one of two orthogonal points are colored~$1$
    and of three pairwise orthogonal points exactly one
    is colored~$1$.
    In public lectures, Conway encouraged the search for small
    KS systems.  
    At the time of writing, the smallest known
    KS system has 31 vectors.  

    Arends, Ouaknine and Wampler have shown that a KS system has at least
    18 vectors, by reducing the problem to the existence of graphs
    with a topological embeddability and non-colorability property.
    Deciding embeddability and the sheer number of graphs on more than~$17$
    vertices, proved the bottleneck in their search.

    Continuing their effort, we restrict our enumeration to smaller class of
    graphs and develop a practical decision procedure for embeddability, to
    establish an improved lower bound of 22.
\end{abstract}
    
\section{Introduction}

% TODO whose idea is this orthogonality graph
TODO

\section{The 33 vector system}
TODO

\section{An improved lower bound}
TODO

\section{Embeddability}
TODO

\section{Acknowledgments}
We wish to thank the following for their generous contribution to the
distributed computation:
    the Digital Security group, Intelligent Systems group
    and the C\&CZ service of the Radboud University;
    Wouter Geraedts and
    Jille Timmermans.

% attribute mckay for answering questions?
\bibliography{main}{}
\bibliographystyle{plain}


\end{document}

% vim: ft=tex.latex
